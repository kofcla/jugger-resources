\subsection*{\centering \LARGE Zum Einpompfen}\addcontentsline{toc}{subsection}{Zum Einpompfen}

\newgame{Einzelduelle}{beliebig}{teilbar durch 2}

% kurze Beschreibung hier
Eine einfache Übung zum Einpomfen. 

\begin{bclogo}[logo=\bcoutil]{Ziel}
Aufwärmen und trainieren spezieller Pomfen und/oder Techniken.
\end{bclogo}

\textbf{\large Regeln}\\
Spieler*innen finden sich in Paaren zusammen und spielen jeweils X Duelle aus. Dann wird Partner gewechselt. 


\newgame{Zombie}{beliebig}{beliebig viele}

% kurze Beschreibung hier
Eine lustige Übung mit vielen möglichen Varianten zum Einpomfen oder für zwischendurch.

\begin{bclogo}[logo=\bcoutil]{Ziel}
Aufwärmen, trainieren der Aufmerksamkeit am Spielfeld.
\end{bclogo}

\textbf{\large Regeln}\\
Spieler*innen verteilen sich über das Spielfeld und versuchen ihre Mitspielenden abzuschlagen. Dabei spielt jede gegen jeden und es gelten Straßenregeln (in den Rücken stechen ist erlaubt). Wer ab ist, darf erst wieder mitspielen, wenn die Person die sie abgeschlagen hat draußen ist. Es wird im Prinzip wie Merkball gespielt.

Das Spiel ist zu ende, wenn nur noch eine Person steht oder nach einer festgesetzten Zeit. \\\newline
\textbf{\large Varianten}
\begin{itemize}
    \item \textbf{But wait - she is not alone ...} Teams aus zwei Personen spielen jeweils gemeinsam (und gewinnen gemeinsam). Zum Teambuilding. 
    \item \textbf{Von überall.} Abgeschlagene Spieler*innen knienen nicht ab, sondern verteilen sich rund um das Spielfeld und können von jeder Seite kommen.
\end{itemize}

\newpage

\newgame{Leader of the hill}{ca. 10 min}{teilbar durch 2}

% kurze Beschreibung hier
Eine Einpomfübung bei denen man sich über 1v1 Duelle an die Spitze hoch kämpft.

\begin{bclogo}[logo=\bcoutil]{Ziel}
Aufwärmen und Einpomfen, viele Duelle gegen viele unterschiedliche Partner*innen.
\end{bclogo}

\textbf{\large Regeln}\\
Es werden zwei Reihen gebildet sodass sich jeweils zwei Duellierpartner*innen gegenüber stehen. Dann werden Duelle gemacht und die gewinnende Person rutscht eine Stufe nach 'oben' (muss definiert werden wo das ist), die verlierende Person rutscht eine Stufe nach 'unten'. Das wird so lange gemacht bis man keine Lust mehr hat oder sich die Reihung der Pomfer*innen nicht mehr viel verändert.

\newgame{Jugger meets Namensspiel}{ca. 10 min}{teilbar durch 2}

% kurze Beschreibung hier
Eine schöne Übung zum Einpomfen mit dem zusätzlichen Vorteil des Namen Lernens und Kommandos Übens.

\begin{bclogo}[logo=\bcoutil]{Ziel}
Üben der Übersicht am Feld, Üben von Kommandos, Namen lernen
\end{bclogo}

\textbf{\large Regeln}\\
Es werden bei dieser Übung \textbf{keine Ketten} verwendet.\\
Die Spielenden werden in zwei Teams aufgeteilt, die sich so gegenüber stehen dass sich immer zwei Personen gegnerischer Teams in einem Duell gegenüber stehen in einer zuvor besprochenen \emph{Spur} von etwa 1.5 bis 2 m die nicht verlassen werden darf.

Ein Team hat gewonnen, wenn alle Gegner*innen knien. Es dürfen alle abgeschlagen werden, nicht nur die Duellpartner*innen solange die Spur nicht verlassen wird.

Wenn man ab ist darf man erst wieder aufstehen, wenn eine noch aktive Person des eigenen Teams einen beim Namen ruft.\\\newline

\textbf{\large Varianten}
\begin{itemize}
    \item Es darf gepinnt werden, dann darf man auch nicht aufstehen wenn der Name gerufen wird. Jetzt ist auch noch auf richtige Calls beim gepinnt werden zu achten.
\end{itemize}

\newpage


\subsection*{\centering \LARGE Zum Rufen}\addcontentsline{toc}{subsection}{Zum Schreien}

\newgame{Rufkreis}{ca. 5 - 10 min}{belibig}
% kurze Beschreibung hier
Kurze Zwischenübung zum Üben laut Kommandos zu rufen (und für Koordination). 

\begin{bclogo}[logo=\bcoutil]{Ziel}
Üben laut zu rufen, Koordination, vertraut machen von Kommandos.
\end{bclogo}

\textbf{\large Regeln}\\

Alle stehen in einem Kreis, es geht im Uhrzeigersinn und jeder ruft ein Kommando und macht eine Bewegung dazu. Abhängig vom Kommando passieren unterschiedliche Dinge.

\textbf{\large Kommandos}

\begin{itemize}
    \item DOWN\newline
    Dazu eine hinknie Bewegung. Dann geht es mit der nächsten Person weiter.

    \item GEMEINSAM\newline
    Zeige dazu auf eine andere Person X. X muss auch GEMEINSAM rufen und dann geht es normal mit der Person neben X weiter. 

    \item HAND oder KOPF\newline
    Währenddessen auf das Körperteil zeigen. Führt zu einem Richtungswechsel.

    \item ROT\newline
    Mit trauriger Bewegung. Nächste Person muss GRÜN rufen mit glücklicher Bewegung.

    \item GRÜN\newline
    Umgekehrtes von Rot. 

    \item DOPPEL\newline
    Gemeinsam mit Zeigen auf eine andere Person und hinknie Bewegung. Diese Person muss auch DOPPEL rufen und runter gehen. Dann geht es mit der Person neben dieser Person weiter. 

    \item WART AUF MICH \newline
    Hebe die Hand und zähle bis 5 Steine, dann geht es weiter mit der nächsten Person.

    \item LINIE HALTEN \newline
    Hake die Arme ein mit den liks und rechten beiden Nachbar*innen. 

    \item MEINER \newline
    Zeige auf die eigene Person mit beiden Armen, nächste Person muss DOWN sagen. Dann geht es normal weiter. 
    
\end{itemize}

\newpage


\subsection*{\centering \LARGE Zum Auspowern}\addcontentsline{toc}{subsection}{Zum Auspowern}

\newgame{Pokemon Power Pomfen}{beliebig}{mindestens 4}

% kurze Beschreibung hier
Eine schöne Übung zum Auspowern am Schluss. 

\begin{bclogo}[logo=\bcoutil]{Ziel}
Auspowern, Duelle unter ständiger Belastung üben
\end{bclogo}

\textbf{\large Regeln}\\
Es werden zwei Reihen aus Pomfer*innen gebildet, wobei sich die ersten Personen der jeweiligen Reihen gegenüber stehen und ein Duell spielen. Die gewinnende Person bleibt vorne, die verlierende geht ans Ende der entsprechenden Reihe. 

\newgame{Pit of Doom}{beliebig}{mindestens drei}

Eine weitere Auspower Übung. 

\begin{bclogo}[logo=\bcoutil]{Ziel}
Auspowern, Duelle unter ständiger Belastung üben
\end{bclogo}

\textbf{\large Regeln}\\
Alle Pompfer*innen stehen im Kreis, eine Pompfer*in in der Mitte. Die Person in der Mitte macht reihum Duelle mit allen Pompfer*innen im Kreis. Dabei wird hauptsächlich auf Geschwindigkeit Wert gelegt. Sobald eine der zwei Duellierenden DOWN ruft, darf die nächste Person ins Duell gehen.
Wenn die Person in der Mitte nicht mehr kann, tauscht sie Platz mit jemand anderem.

\newgame{Kartenlauf}{beliebig}{2 Teams zu je 3 bis 10 Leuten}

% kurze Beschreibung hier
Von zwei Teams sprinten jeweils eine Person von einer Startlinie zu einem Ende und deckt Spielkarten auf. Ist es die Karte des eigenen Teams, bringt sie diese zurück und das Team, das alle gesammelt hat, gewinnt.

\begin{bclogo}[logo=\bcoutil]{Ziel}
Sprinten und stoppen, aber auch Kommunikation mit einem Team.
\end{bclogo}

\textbf{\large Regeln}\\
An einem Ende einer Strecke werden Spielkarten aufgelegt. Wichtig ist, dass es gleich viele schwarze wie weiße sind und im Idealfall mehr Karten jeder Farbe als die Größe eines Teams. Die Anzahl der Karten pro Farbe muss denn Teams mitgeteilt werden. Die Karten sind gemischt und den Teams werden schwarz und rot zugeteilt. Pro Team läuft immer eine Person die Länge zu den Karten und darf \emph{eine} Karte aufdecken, ist es die Farbe des eigenen Teams darf sie mitgenommen werden, wenn nicht, muss sie zurückgelegt werden. Dann läuft man zurück und klatscht mit dem nächsten Teammitglied ab und diese Person läuft als Nächstes. Das Team, das als Erstes die zuvor verkündete Maximalzahl an Karten hat, kann, nachdem sie diese Karten zurückgebracht hat, "Ende" rufen und hat das Spiel gewonnen.


\subsection*{\centering \LARGE Training}\addcontentsline{toc}{subsection}{Training}




\newgame{Läuferweg}{beliebig}{5er Teams}

Übung zu Laufwegen und Malverteidigung

\begin{bclogo}[logo=\bcoutil]{Ziel}
Läufer Sollen lernen Pins zu lösen ohne abgeschlagen zu werden.
\end{bclogo}

\textbf{\large Regeln}\\
Zwei Personen aus dem Team vom Läufer sind in der Nähe des Mals gepinnt. Der Läufer ist aus dem Team der gepinnten Spieler und versucht den Jug zu stecken. Die Pinnenden Spieler gewinnen, wenn sie beide Pins halten und den Jug kontrollieren. Gewinnt das Team mit Läufer, werden die Pins näher zum Mal versetzt. Andernfalls werden sie weiter weg verschoben.



\newgame{Pinrad}{beliebig}{3er Teams}

Übung zum Pinnen

\begin{bclogo}[logo=\bcoutil]{Ziel}
Übung um Pins zu verteidigen und Doppelpins zu halten
\end{bclogo}

\textbf{\large Regeln}\\
Eine Person ist am Anfang gepint und eine pint. Die dritte Person hat die Aufgabe den Pin frei zuspielen. Wenn sie es schafft, wird sie die pinnende Person und die Person die aufsteht muss nun versuchen den Pin frei zuspielen. Wenn sie es nicht schafft, muss die pinnende Person versuchen einen Doppelpin zu halten und wechselt aber den Pin möglichst nach 5 Steinen auf die Person, die sie als letztes Abgeschlagen hat. 


\newgame{Bamberger Offensiv/Defensivübung}{beliebig}{Vielfache von drei}

Duellübung zum Techniktraining.

\begin{bclogo}[logo=\bcoutil]{Ziel}
Eine Duellübung um offensives und defensives Spiel gezielt zu trainieren und die Kondition zu steigern
\end{bclogo}

\textbf{\large Regeln}\\
Die Pompfer*innen bilden 3er-Gruppen. Zwei Member jeder Gruppe stehen ca. 5-10m auseinander, diese Startposition kann mit Hütchen o.Ä. markiert werden. Der dritte Member startet auf einer beliebigen Seite und attackiert das von ihm weiter entfernt stehende Gruppenmitglied. Der*die Sieger*in rennt dann die Strecke zurück und nimmt das Duell mit dem verbliebenen Gruppenmitglied. Der Sieger rennt wiederum auf die andere Seite usw.. Die verteidigenden Gruppenmitglieder versuchen defensiv zu spielen und Raum aufzugeben, während die rennenden Spieler versuchen, dass Duell offensiv zu entscheiden.




\newgame{Mal verteidigen}{beliebig}{3er Teams}

Koordinationsübung für Läufer und Pompfer

\begin{bclogo}[logo=\bcoutil]{Ziel}
Läufer und Pompfer sollen zusammenarbeiten um ein möglichst große Gefahr darzustellen.
\end{bclogo}

\textbf{\large Regeln}\\
Läufer und Pompfer attackieren gemeinsam das Mal, welches von einem Pompfer verteidigt wird. Sie gewinnen, wenn sie den Jug stecken. Der Pompfer gewinnt, wenn er den Jug in einem Pin kontrolliert. Als Verteidigung ist auch eine Kette möglich. In dem Fall gewinnt die Kette, wenn sie beide abgeschlagen hat.

\newgame{Chaos Ketten}{beliebig}{2 Teams á 4 + 2}

Zwei unzugeordnete Ketten versuchen zu verhindern dass ein Team einen Jugg steckt. 


\begin{bclogo}[logo=\bcoutil]{Ziel}
Taktische Übersicht (va bzgl. Jugg besitz) soll gestärkt werden. 
\end{bclogo}

\textbf{\large Regeln}\\
Zwei Teams á 4 Personen (ohne Ketten) versuchen ganz normal einen Jugg zu stecken beim gegnerischen Team. Auf der Mittellinie stehen zwei Ketten, die keinem Team zugeordnet sind und verhindern müssen dass ein Jugg gesteckt wird. 



\newgame{2v1+L Duell}{min. 5 min}{4}

2 v 1 + Läufer Duell. 


\begin{bclogo}[logo=\bcoutil]{Ziel}
Übung von 2v1 Pompfer*innen und taktischer Nutzen von Läufer*innen in Unterzahl Situationen. 
\end{bclogo}

\textbf{\large Regeln}\\
Auf einer Hallenseite zwei Pompfer*innen verteidigen das Mal. Von der anderen Hallenseite ein*e Pompfer*in und ein*e Läufer*in mit Jugg versuchen einen Jugg zu stecken.

\newpage

\newgame{Endlos Duell und Pinwechsel}{beliebig}{mindestens 3 Spielende}

Übung für Duelle und Pinwechsel in einer fast beliebig großen Gruppe.

\begin{bclogo}[logo=\bcoutil]{Ziel}

Auspowern, Duelle üben, Pinwechsel üben, Namen lernen

\end{bclogo}

\textbf{\large Regeln}\\

Es finden n Duelle gleichzeitig statt, während m Personen vom Rand aus zuschauen. Sobald ein Duell entschieden ist, pinnt der*die Gewinner*in den*die Duellgegner*in. Eine der zuschauenden Personen sagt dann einen Pinwechsel an (Kommando inkl. Name der pinnenden Person). Nach dem erfolgreichen Pinwechsel läuft die ehemals pinnende Person zu den Zuschauern und die gepinnte Person duelliert sich mit der Person, die den Pin übernommen hat.

\newgame{Kettenaufbautraining}{kurz}{1v1}

Training zur Kettendistanz und Kette/Antikette. 


\begin{bclogo}[logo=\bcoutil]{Ziel}
Kennenlernen von Kettendistanzen und den Vorteilen/Nachteilen für Kette und Antikette die damit assoziiert sind. 
\end{bclogo}

\textbf{\large Regeln}\\
Übung in drei Teilen:

\begin{enumerate}
    \item halbe 
\end{enumerate}
