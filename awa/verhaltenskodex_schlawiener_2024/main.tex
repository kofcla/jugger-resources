\documentclass{assignment}
\usepackage{tabularx}
%\usepackage[T1]{fontenc}
%\usepackage{ascii}
\usepackage{eurosym}
\usepackage{amsmath}
\usepackage{amssymb}
\usepackage{multicol}
\usepackage{dashrule}
\usepackage{pifont}


\usepackage[scaled=1.05,proportional,lightcondensed]{zlmtt}

\setlength{\parindent}{0pt}
\begin{document}
\pagestyle{empty}

\begin{table}[h!]
    \centering
    \begin{tabularx}{\textwidth}{X  r}
        \footnotesize \url{https://juggerschlawiener.com/}&\footnotesize \url{jugger.vienna@gmail.com}
    \end{tabularx}
    
    \vspace{30pt}

\end{table}

\assignmentTitle{https://juggerschlawiener.com/}{jugger.vienna@gmail.com}{assets/logo3.4.png}{SchlaWiener}{Verhaltenskodex}



Liebe SchlaWiener, damit sich alle wohlfühlen bei uns haben wir hier einen Verhaltenskodex für ein gutes Miteinander. Bitte durchlesen und darüber nachdenken. 

\rule{\textwidth}{0.4pt}
%\vspace{25pt}

\begin{itemize}
    \item Jugger ist ein Gemeinschaftssport: Wir helfen zusammen und respektieren einander gegenseitig.
    \item Die Sicherheit und das Wohlbefinden jeder Person hat beim Jugger den höchsten Stellenwert. Man ist erst dann ein*e echte*r Jugger Spieler*in, wenn man niemanden mit seinem Spielstil verletzt oder gefährdet.
    \item Wir spielen „sportlich“. Das heißt wir halten uns an die Spielregeln und sind fair.
    \item Fehler sind okay. Wir müssen nicht böse werden, wenn jemand anders einen Fehler macht. Egal ob im eigenen Team oder bei den Gegnern.
    \item Jugger ist ein inklusiver Sport. Das heißt wir haben mixed gender Teams und das ist gut so.
    \item Wir diskriminieren niemanden: Jede Person hat bei uns Platz, egal welches Geschlecht, welche Hautfarbe, welche Religion, …
    \item Wir schauen darauf, dass es jedem gut geht. Wenn wir bemerken, dass jemand ein Problem hat, versuchen wir zu helfen und sagen es einem/r der Trainer*innen. 
    \item Es ist in Ordnung nicht alles zu können. Wenn jemand eine Pause braucht, dann machen wir eine Pause. Gesundheit ist wichtiger als Erfolg im Spiel.
    \item Wenn ich mich unwohl fühle wegen irgendetwas, kann ich mit einem/r der Trainer*innen reden und sie werden versuchen, eine Lösung zu finden. 
    \item Wir achten auf die Grenzen und Gefühle unsere Mitspielenden.
    \item Wir schauen darauf, dass sich alle wohl fühlen.
\end{itemize}

\end{document}
